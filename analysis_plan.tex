\documentclass[]{article}
\usepackage{lmodern}
\usepackage{amssymb,amsmath}
\usepackage{ifxetex,ifluatex}
\usepackage{fixltx2e} % provides \textsubscript
\ifnum 0\ifxetex 1\fi\ifluatex 1\fi=0 % if pdftex
  \usepackage[T1]{fontenc}
  \usepackage[utf8]{inputenc}
\else % if luatex or xelatex
  \ifxetex
    \usepackage{mathspec}
  \else
    \usepackage{fontspec}
  \fi
  \defaultfontfeatures{Ligatures=TeX,Scale=MatchLowercase}
\fi
% use upquote if available, for straight quotes in verbatim environments
\IfFileExists{upquote.sty}{\usepackage{upquote}}{}
% use microtype if available
\IfFileExists{microtype.sty}{%
\usepackage[]{microtype}
\UseMicrotypeSet[protrusion]{basicmath} % disable protrusion for tt fonts
}{}
\PassOptionsToPackage{hyphens}{url} % url is loaded by hyperref
\usepackage[unicode=true]{hyperref}
\hypersetup{
            pdftitle={Old Faithful Analysis Plan},
            pdfauthor={Rob Arbon},
            pdfborder={0 0 0},
            breaklinks=true}
\urlstyle{same}  % don't use monospace font for urls
\usepackage[margin=1in]{geometry}
\usepackage{graphicx,grffile}
\makeatletter
\def\maxwidth{\ifdim\Gin@nat@width>\linewidth\linewidth\else\Gin@nat@width\fi}
\def\maxheight{\ifdim\Gin@nat@height>\textheight\textheight\else\Gin@nat@height\fi}
\makeatother
% Scale images if necessary, so that they will not overflow the page
% margins by default, and it is still possible to overwrite the defaults
% using explicit options in \includegraphics[width, height, ...]{}
\setkeys{Gin}{width=\maxwidth,height=\maxheight,keepaspectratio}
\IfFileExists{parskip.sty}{%
\usepackage{parskip}
}{% else
\setlength{\parindent}{0pt}
\setlength{\parskip}{6pt plus 2pt minus 1pt}
}
\setlength{\emergencystretch}{3em}  % prevent overfull lines
\providecommand{\tightlist}{%
  \setlength{\itemsep}{0pt}\setlength{\parskip}{0pt}}
\setcounter{secnumdepth}{0}
% Redefines (sub)paragraphs to behave more like sections
\ifx\paragraph\undefined\else
\let\oldparagraph\paragraph
\renewcommand{\paragraph}[1]{\oldparagraph{#1}\mbox{}}
\fi
\ifx\subparagraph\undefined\else
\let\oldsubparagraph\subparagraph
\renewcommand{\subparagraph}[1]{\oldsubparagraph{#1}\mbox{}}
\fi

% set default figure placement to htbp
\makeatletter
\def\fps@figure{htbp}
\makeatother


\title{Old Faithful Analysis Plan}
\author{Rob Arbon}
\date{09/11/2020}

\begin{document}
\maketitle

\section{Hypothesis}\label{hypothesis}

Based on previous exploratory data analysis (found in
\texttt{old\_faithful\_eda.pdf}) we propose the following hypothesis:

\emph{There are 2 distinct types of eruption from Old Faithful: 1) short
frequent eruptions, 2) long infrequent eruptions.}

We will use the dataset \texttt{oldfaithful} available from the
\texttt{R} package \texttt{datasets}. See
\href{https://stat.ethz.ch/R-manual/R-devel/library/datasets/html/faithful.html}{here}
for the documentation.

\section{Analysis plan}\label{analysis-plan}

We will fit two Gaussian mixture models with 1 and 2 components. The
number of components will be deteremined by the model with the highest
Integrated Complete Data Likelihood Criterion (ICL). This criterion is
similar to the
\href{https://en.wikipedia.org/wiki/Bayesian_information_criterion}{Bayesian
information criterion} but is more suited to classification problems.

The true number of components will be given by the model with the
highest Integrated Complete Data Likelihood (ICL). The model will be fit
in the R package \texttt{RMixmod}.

\end{document}
